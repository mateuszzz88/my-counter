\documentclass[a4paper]{article}
\usepackage{polski}
\usepackage[utf8]{inputenc}
\usepackage{latexsym}
\usepackage{graphicx}
\usepackage{marvosym}
\usepackage[vcentering,dvips]{geometry}
\geometry{papersize={210mm,270mm},total={161mm,231mm}}
\usepackage{setspace}

\author{Przemysław Ogorzałek\\Mateusz Lewicki}
\title{ZPR\\Projekt\\Klasa bazowa do obliczeń\\Dokumentacja końcowa}

\begin{document}
\maketitle
\section{Opis zagadnienia}
Celem projektu jest stworzenie klasy bazowej zapewniającej interfejs do~obsługi złożonych czasowo obliczeń prowadzonych przez komputer.\\
Zadaniem klasy jest dostarczenie funkcjonalności umożliwiających przerywanie i~wznawianie obliczeń, możliwość zapisu do pliku postępów pracy zarówno na polecenie użytkownika jak i~co~pewien interwał czasowy. 
Zapisane wyniki można odczytać celem kontynuowania obliczeń od danego punktu.\\
Użytkownik końcowy powinien jak najmniej ingerować w~kod klasy, aby przystosować ją do swoich celów.
W~założeniu nie powinien zmieniać więcej niż jednej lub dwóch metod.\\
Nie jest podana ani liczba ani typ zmiennych, które są wykorzystane do zapisu wyników.
Decyduje o~tym użytkownik końcowy.
\section{Interfejs programu}
\subsection{Implementacja}
Założeniem które przyświeca konstrukcji klasy jest ograniczenie konieczności ingerencji w~kod przez użtykownika do niezbędnego minimum.\\
Aby klasa działała poprawnie, użytkownik musi jedynie dostarczyć strukturę, w~której zapisywane będą wyniki, oraz metodę wykonującą kolejne kroki obliczeń.
W~szczególnych przypadkach, gdy działania nie da się podzielić na mniejsze, kolejno wykonywane identyczne kroki, może zajść konieczność wprowadzenia zmian do metody zarządzającej obliczeniami.\\
~\\
Struktura zapisująca wyniki jest w~całości stworzona przez użytkownika końcowego. 
Do klasy przeprowadzającej obliczenia przekazywana jest ona jako parametr szablonu. \\
Wewnątrz niej użytkownik może umieścić dowolną liczbę dowolnego typu zmiennych. 
Odwoływać się do nich będzie jedynie wewnątrz funkcji prowadzącej obliczenia, którą sam musi zaimplementować.\\
Jedynym ograniczeniem nałożonym na niego jest konieczność umieszczenia wewnątrz klasy funkcji umożliwiającej serializację.
Wewnątrz tej funkcji muszą znaleźć się odwołania do wszystkich zmiennych, których wartość jest konieczna do późniejszego kontynuowania obliczeń na podstawie danych wczytanych z~pliku.
Jeśli zmienne są obiektami klas lub struktur zdefiniowanych przez użytkownika, to wewnątrz tych klas również muszą znaleźć się funkcje odpowiedzialne za serializację.\\
Wymagany jest również konstruktor kopiujący.\\
~\\
Przykład:\\
Wyniki zapisywane są w~obiekcie struktury \texttt{DataStruct}, na dwóch zmiennych typu \texttt{int} i~obiekcie struktury \texttt{HelpStruct}, który ma trzy zmienne typu \texttt{bool}, z~których dwie przechowują wynik obliczeń, a~trzecia jest używana do innych celów i~zapisywanie jej wartości do pliku jest zbędne.\\
Implementacja takich struktur wygląda następująco:
\begin{verbatim}
struct HelpStruct
{
  bool istotna1_, istotna2_, nieistotna_;
  HelpStruct(bool a, bool b, bool c): istotna1_(a), istotna2_(b), nieistotna_(c) {}
  HelpStruct(const HelpStruct& hs): istotna1_(hs.istotna1_), istotna2_(hs.istotna2_), 
                                    nieistotna_(hs.nieistotna_) {};

  template<class Archive>
  void serialize(Archive & ar, const unsigned int version)
  {
    ar & istotna1_;
    ar & istotna2_;
  }
  
};

struct DataStruct
{
  int x_,y_;
  HelpStruct hs_;
  DataStruct(int a, int b, HelpStruct hest): x_(a), y_(b), hs_(hest) {};
  DataStruct(const DataStruct& ds): x_(ds.x_), y_(ds.y_), hs_(ds.hs_) {};

  template<class Archive>
  void serialize(Archive & ar, const unsigned int version)
  {
    ar & x_;
    ar & y_;
    ar & hs_;
  }
};

\end{verbatim}
Taka implementacja zapewnia zarówno możliwość zapisywania wartości odpowiednich składowych do pliku, jak również ich odczytania.
Funkcja serializacji ma stałą strukturę i~nie wymaga niczego poza wyszczególnieniem zmiennych w~sposób przedstawiony powyżej.
Jest ona później wywoływana przez program w~celu zapisania lub odczytania wartości zmiennych.\\
Jeśli w~strukturze użytkownika znajdują się wskaźniki, należy podać w~funkcji serializacji ich wartość, przykładowo:
\begin{verbatim}
ar & *wskaznik;
\end{verbatim}
Składowa przechowująca obiekt struktury podanej przez użytkownika nazywa się \texttt{data}.\\
Dla typów wbudowanych nie jest konieczne tworzenie funkcji serializacji, więc jeśli do zapisu wyników wystarczy nam na przykład liczba typu \texttt{int}, tworzenie osobnej struktury jest zbędne.\\
~\\
Aby przeprowadzić obliczenia konieczna jest również własna implementacja metody \texttt{Calculate()} będącej częścią klasy \texttt{CounterBase}. 
Dokonuje się tego poprzez mechanizm szablonów.\\
~\\
Przykład dla struktury \texttt{DataStruct}:
\begin{verbatim}
//Deklaracja
template<>
CounterBase<DataStruct>::Calculate();
//...
//Implementacja
template<>
CounterBase<DataStruct>::Calculate()
{
 //Jakieś obliczenia
 //Do składowych struktury odwołujemy się następująco
 //data.nazwa_pola
 //O zakończeniu obliczeń informujemy zmieniając wartość zmiennej
 //this->stop
 //na przeciwną

} 
\end{verbatim}
Aby implementacja działała poprawnie obliczenia muszą być podzielone na małe etapy.
Każde wywołanie funkcji \texttt{Calculate()} to przejście przez kolejny etap.\\
Zapisanie postępów pracy dokonuje się pomiędzy kolejnymi wywołaniami funkcji, co pewien określony czas.
Odstęp między kolejnymi zapisami zachowany jest zatem z~dokładnością do czasu trwania pojedynczego wywołania funkcji.\\
Po zakończeniu obliczeń należy zmienić wartość pola klasy \texttt{stop} należącego do klasy \texttt{CounterBase} na przeciwną.
Wynik końcowy zostanie wówczas zapisany, a~obliczenia zakończą się.\\
~\\
W~przypadku, gdzy podzielenie obliczeń na wiele wywołań funkcji \texttt{Calculate()} jest niemożliwe, konieczna jest ponowna implementacja funkcji \texttt{doCalculate()}, zarządzającej obliczeniami i~zapisywaniem wyników. 
To jednak zaleca się jedynie doświadczonym programistom, ponieważ wymaga to wnikliwej analizy kodu klasy.
\subsection{Kompilacja}
Do kodu źródłowego klasy dołączone są przykłądowe pliki sterujące programów \texttt{make} i~\texttt{Scons}.\\
Aby kompliacja przebiegła pomyślnie kompilator musi mieć dostęp do bibliotek \texttt{boost}, w~szczególności:
\begin{itemize}
\item \texttt{boost::serialization}
\item \texttt{boost::thread} 
\item \texttt{boost::bind}
\item \texttt{boost::date\_time}
\end{itemize}
\subsection{Prowadzenie obliczeń}
Użytkownik komunikuje się z~programem poprzez wpisywanie w~konsoli odpowiednich komend, jak i~poprzez sygnały systemowe. 
Na początku działania programu lub po wpisaniu polecenia \texttt{help} wyświetla się lista dostępnych poleceń i~sygnałów\footnote{W~systemach opartycha na UNIX sygnały systemowe wysyła się poleceniem\\\texttt{kill -SYGNAŁ PID}\\gdzie \texttt{SYGNAŁ} to nazwa sygnału, a \texttt{PID} to numer identyfikacyjny procesu, w ramach którego działa program.\\Listę procesów wraz z~numerami można wyświetlić poleceniem\\\texttt{ps -a}}.\\
Obliczenia mogą być prowadzone bądź w~wątku głównym, bądź w~utworzonym specjalnie do tego celu.

\section{Działanie klasy bazowej}
W~zależności od potrzeb wczytywany jest plik z~ostatnio zapisanym wynikiem lub obliczenia zaczynane są od początku.
Co pewien czas postęp obliczeń jest zapisywany. 
Zapisu dokonać można również wysyłając sygnał \texttt{SIGUSR1}.\\
Aby zapuzować obliczenia wysyłamy sygnał \texttt{SIGTSTP}\footnote{Zwykle można tego dokonać poprzez skrót klawiaturowy \texttt{CTRL+z}}. 
Nawet jeśli obliczenia są zapauzowane, aktualny wynik nadal jest zapisywany.
Aby wznowić działanie programu, albo wysyłamy ponownie \texttt{SIGTSTP}, albo wysyłamy sygnał \texttt{SIGCONT}.
Wysłanie \texttt{SIGCONT} jeśli program działa jest ignorowane.\\
Aby przerwać obliczenia wpisujemy komendę stop, lub wysyłamy jeden z~sygnałów: \texttt{SIGINT}\footnote{Zwykle można tego dokonać poprzez skrót klawiaturowy \texttt{CTRL+c}} lub \texttt{SIGTERM}.
Wynik obliczeń zostanie wtedy zapisany, a~wątki prowadzące obliczenia zakończą się.\\
~\\
Aby przeprowadzać kolejne kroki obliczeń, funkcja \texttt{doCalculate()} odczytuje wartość poszczególnych flag i~w~zależności od ich wartości zapisuje wynik, wywołuje funkcję \texttt{Calculate()} lub przerywa wątek.
\section{Napotkane problemy}
\subsection{Serializacja}
Aby klasa poprawnie realizowała założenia projektu musi spełnić dwa przeciwstawne warunki: umożliwić zapis dowolnego typu danych do pliku i~nałożyć na użytkownika końcowego jak najmniej obowiązków związanych z~implementacją klasy.\\
W~trywialnym przypadku wymagałoby to zdefiniowania dla struktury z~danymi operatorów strumieniowego wejścia i~wyjścia, co mogłoby być uciążliwe, zwłaszcza dla skomplikowanych klas.
Wymagałoby to również dokładnego przemyślenia jak wyglądać powinien plik przechowujący dane.\\
Kompromisem jest tu wprowadzenie serializacji, to jest zamiany danych na postać bajtową i~zapisanie jej do pliku, a~następnie powtórne jej odczytanie.\\
Mechanizm ten wymaga jedynie zimplementowania prostej funkcji publicznej w~strukturze przechowującej dane.\\
W~przeciwieństwie np. do języka Java, C++ nie zapewnia jednak takiej możliwości, ponieważ nie jest językiem czysto obiektowym.
Rozwiązaniem jest tu biblioteka \texttt{boost::serialization}, która wprowadza odpowiednie mechanizmy zapisu i~odczytu pliku.
Użytkownik końcowy nie musi wiedzieć, w~jakim formacie zapisano dane, ani jak dokładnie działa ten mechanizm. 
Jego rola ogranicza się do wylistowania zmiennych, które mają być zapisane. 
\subsection{Komunikacja}
Aby możliwe było pauzowanie i~przerywanie obliczeń konieczne jest zapewnienie użytkownikowi możliwości przesłania do programu odpowiedniej informacji.
Język C++, w~przeciwieństwie np. do Javy nie jest językiem zapewniającym obsługę zdarzeń i~nie przewiduje stosowania listenerów.\\
Aby zapewnić porządaną funkcjonalność wykorzystujemy sygnały systemu UNIX, zapisane w~bibliotece \texttt{signal.h}.
W~dołączonych do klasy plikach znajdują się funkcje globalne przechwytujące odpowiednie sygnały i~zmieniające wartość odpowiednich zmiennych.
Mechanizm ten pozwala uniknąć niewygodnego w~tym przypadku mechanizmu funkcji i~zmiennych statycznych, a~także pozwala użytkownikowi w~łatwy sposób stworzyć własne funkcje sterujące, jeśli pojawią się wymagania inne od tych zawartych w~założeniach projektu.\\
Wykorzystywane sygnały systemowe to:
\begin{itemize}
\item \texttt{SIGINT} - przerywa obliczenia i~zapisuje wynik do pliku
\item \texttt{SIGTERM} - przerywa obliczenia i~zapisuje wynik do pliku
\item \texttt{SIGTSTP} - pauzuje i~wznawia obliczenia
\item \texttt{SIGCONT} - wznawia zapauzowane obliczenia
\item \texttt{SIGUSR1} - zapisuje wynik obliczeń do pliku
\end{itemize}
Mechanizm ten pozwala w~łatwy sposób przesyłać odpowiednie informacje i~pozwala wykorzystać narzędzia dostarczane przez system operacyjny, co pozwala ograniczyć rozmiar kodu programu.
Wadą tego mechanizmu jest niemożność jego wykorzystania na platformie Windows. 
\section{Podsumowanie}
Zaprojektowana i~stworzona klasa spełnia wszystkie wstępne założenia projektowe.\\
Zapewnia odpowiednie funkcjonalności, takie jak zapisywanie i~odczytywanie postępów pracy, pauzowanie, czy wznawianie obliczeń.
Wykorzystuje przy tym najnowsze rozwiązania programistyczne, zawarte w~bibliotece \texttt{boost} i~narzędzia systemu operacyjnego w~postaci sygnałów.\\
Aby wykorzystać ją do swoich celów użytkownik musi jedynie zaimplementować strukturę przechowującą dane wynikowe zawierającą funkcję serializacji i~funkcję wykonującą kolejne kroki obliczeń.\\
Wszystkie te cechy sprawiają, że można niewielkim nakładem pracy wykorzystać ją do zabezpieczenia swoich obliczeń przez niespodziewaną utratą wyników.

\end{document}
